\documentclass[11pt]{article}
\usepackage[UTF8]{ctex}

    \usepackage[breakable]{tcolorbox}
    \usepackage{parskip} % Stop auto-indenting (to mimic markdown behaviour)
    

    % Basic figure setup, for now with no caption control since it's done
    % automatically by Pandoc (which extracts ![](path) syntax from Markdown).
    \usepackage{graphicx}
    % Keep aspect ratio if custom image width or height is specified
    \setkeys{Gin}{keepaspectratio}
    % Maintain compatibility with old templates. Remove in nbconvert 6.0
    \let\Oldincludegraphics\includegraphics
    % Ensure that by default, figures have no caption (until we provide a
    % proper Figure object with a Caption API and a way to capture that
    % in the conversion process - todo).
    \usepackage{caption}
    \DeclareCaptionFormat{nocaption}{}
    \captionsetup{format=nocaption,aboveskip=0pt,belowskip=0pt}

    \usepackage{float}
    \floatplacement{figure}{H} % forces figures to be placed at the correct location
    \usepackage{xcolor} % Allow colors to be defined
    \usepackage{enumerate} % Needed for markdown enumerations to work
    \usepackage{geometry} % Used to adjust the document margins
    \usepackage{amsmath} % Equations
    \usepackage{amssymb} % Equations
    \usepackage{textcomp} % defines textquotesingle
    % Hack from http://tex.stackexchange.com/a/47451/13684:
    \AtBeginDocument{%
        \def\PYZsq{\textquotesingle}% Upright quotes in Pygmentized code
    }
    \usepackage{upquote} % Upright quotes for verbatim code
    \usepackage{eurosym} % defines \euro

    \usepackage{iftex}
    \ifPDFTeX
        \usepackage[T1]{fontenc}
        \IfFileExists{alphabeta.sty}{
              \usepackage{alphabeta}
          }{
              \usepackage[mathletters]{ucs}
              \usepackage[utf8x]{inputenc}
          }
    \else
        \usepackage{fontspec}
        \usepackage{unicode-math}
    \fi

    \usepackage{fancyvrb} % verbatim replacement that allows latex
    \usepackage{grffile} % extends the file name processing of package graphics
                         % to support a larger range
    \makeatletter % fix for old versions of grffile with XeLaTeX
    \@ifpackagelater{grffile}{2019/11/01}
    {
      % Do nothing on new versions
    }
    {
      \def\Gread@@xetex#1{%
        \IfFileExists{"\Gin@base".bb}%
        {\Gread@eps{\Gin@base.bb}}%
        {\Gread@@xetex@aux#1}%
      }
    }
    \makeatother
    \usepackage[Export]{adjustbox} % Used to constrain images to a maximum size
    \adjustboxset{max size={0.9\linewidth}{0.9\paperheight}}

    % The hyperref package gives us a pdf with properly built
    % internal navigation ('pdf bookmarks' for the table of contents,
    % internal cross-reference links, web links for URLs, etc.)
    \usepackage{hyperref}
    % The default LaTeX title has an obnoxious amount of whitespace. By default,
    % titling removes some of it. It also provides customization options.
    \usepackage{titling}
    \usepackage{longtable} % longtable support required by pandoc >1.10
    \usepackage{booktabs}  % table support for pandoc > 1.12.2
    \usepackage{array}     % table support for pandoc >= 2.11.3
    \usepackage{calc}      % table minipage width calculation for pandoc >= 2.11.1
    \usepackage[inline]{enumitem} % IRkernel/repr support (it uses the enumerate* environment)
    \usepackage[normalem]{ulem} % ulem is needed to support strikethroughs (\sout)
                                % normalem makes italics be italics, not underlines
    \usepackage{soul}      % strikethrough (\st) support for pandoc >= 3.0.0
    \usepackage{mathrsfs}
    

    
    % Colors for the hyperref package
    \definecolor{urlcolor}{rgb}{0,.145,.698}
    \definecolor{linkcolor}{rgb}{.71,0.21,0.01}
    \definecolor{citecolor}{rgb}{.12,.54,.11}

    % ANSI colors
    \definecolor{ansi-black}{HTML}{3E424D}
    \definecolor{ansi-black-intense}{HTML}{282C36}
    \definecolor{ansi-red}{HTML}{E75C58}
    \definecolor{ansi-red-intense}{HTML}{B22B31}
    \definecolor{ansi-green}{HTML}{00A250}
    \definecolor{ansi-green-intense}{HTML}{007427}
    \definecolor{ansi-yellow}{HTML}{DDB62B}
    \definecolor{ansi-yellow-intense}{HTML}{B27D12}
    \definecolor{ansi-blue}{HTML}{208FFB}
    \definecolor{ansi-blue-intense}{HTML}{0065CA}
    \definecolor{ansi-magenta}{HTML}{D160C4}
    \definecolor{ansi-magenta-intense}{HTML}{A03196}
    \definecolor{ansi-cyan}{HTML}{60C6C8}
    \definecolor{ansi-cyan-intense}{HTML}{258F8F}
    \definecolor{ansi-white}{HTML}{C5C1B4}
    \definecolor{ansi-white-intense}{HTML}{A1A6B2}
    \definecolor{ansi-default-inverse-fg}{HTML}{FFFFFF}
    \definecolor{ansi-default-inverse-bg}{HTML}{000000}

    % common color for the border for error outputs.
    \definecolor{outerrorbackground}{HTML}{FFDFDF}

    % commands and environments needed by pandoc snippets
    % extracted from the output of `pandoc -s`
    \providecommand{\tightlist}{%
      \setlength{\itemsep}{0pt}\setlength{\parskip}{0pt}}
    \DefineVerbatimEnvironment{Highlighting}{Verbatim}{commandchars=\\\{\}}
    % Add ',fontsize=\small' for more characters per line
    \newenvironment{Shaded}{}{}
    \newcommand{\KeywordTok}[1]{\textcolor[rgb]{0.00,0.44,0.13}{\textbf{{#1}}}}
    \newcommand{\DataTypeTok}[1]{\textcolor[rgb]{0.56,0.13,0.00}{{#1}}}
    \newcommand{\DecValTok}[1]{\textcolor[rgb]{0.25,0.63,0.44}{{#1}}}
    \newcommand{\BaseNTok}[1]{\textcolor[rgb]{0.25,0.63,0.44}{{#1}}}
    \newcommand{\FloatTok}[1]{\textcolor[rgb]{0.25,0.63,0.44}{{#1}}}
    \newcommand{\CharTok}[1]{\textcolor[rgb]{0.25,0.44,0.63}{{#1}}}
    \newcommand{\StringTok}[1]{\textcolor[rgb]{0.25,0.44,0.63}{{#1}}}
    \newcommand{\CommentTok}[1]{\textcolor[rgb]{0.38,0.63,0.69}{\textit{{#1}}}}
    \newcommand{\OtherTok}[1]{\textcolor[rgb]{0.00,0.44,0.13}{{#1}}}
    \newcommand{\AlertTok}[1]{\textcolor[rgb]{1.00,0.00,0.00}{\textbf{{#1}}}}
    \newcommand{\FunctionTok}[1]{\textcolor[rgb]{0.02,0.16,0.49}{{#1}}}
    \newcommand{\RegionMarkerTok}[1]{{#1}}
    \newcommand{\ErrorTok}[1]{\textcolor[rgb]{1.00,0.00,0.00}{\textbf{{#1}}}}
    \newcommand{\NormalTok}[1]{{#1}}

    % Additional commands for more recent versions of Pandoc
    \newcommand{\ConstantTok}[1]{\textcolor[rgb]{0.53,0.00,0.00}{{#1}}}
    \newcommand{\SpecialCharTok}[1]{\textcolor[rgb]{0.25,0.44,0.63}{{#1}}}
    \newcommand{\VerbatimStringTok}[1]{\textcolor[rgb]{0.25,0.44,0.63}{{#1}}}
    \newcommand{\SpecialStringTok}[1]{\textcolor[rgb]{0.73,0.40,0.53}{{#1}}}
    \newcommand{\ImportTok}[1]{{#1}}
    \newcommand{\DocumentationTok}[1]{\textcolor[rgb]{0.73,0.13,0.13}{\textit{{#1}}}}
    \newcommand{\AnnotationTok}[1]{\textcolor[rgb]{0.38,0.63,0.69}{\textbf{\textit{{#1}}}}}
    \newcommand{\CommentVarTok}[1]{\textcolor[rgb]{0.38,0.63,0.69}{\textbf{\textit{{#1}}}}}
    \newcommand{\VariableTok}[1]{\textcolor[rgb]{0.10,0.09,0.49}{{#1}}}
    \newcommand{\ControlFlowTok}[1]{\textcolor[rgb]{0.00,0.44,0.13}{\textbf{{#1}}}}
    \newcommand{\OperatorTok}[1]{\textcolor[rgb]{0.40,0.40,0.40}{{#1}}}
    \newcommand{\BuiltInTok}[1]{{#1}}
    \newcommand{\ExtensionTok}[1]{{#1}}
    \newcommand{\PreprocessorTok}[1]{\textcolor[rgb]{0.74,0.48,0.00}{{#1}}}
    \newcommand{\AttributeTok}[1]{\textcolor[rgb]{0.49,0.56,0.16}{{#1}}}
    \newcommand{\InformationTok}[1]{\textcolor[rgb]{0.38,0.63,0.69}{\textbf{\textit{{#1}}}}}
    \newcommand{\WarningTok}[1]{\textcolor[rgb]{0.38,0.63,0.69}{\textbf{\textit{{#1}}}}}
    \makeatletter
    \newsavebox\pandoc@box
    \newcommand*\pandocbounded[1]{%
      \sbox\pandoc@box{#1}%
      % scaling factors for width and height
      \Gscale@div\@tempa\textheight{\dimexpr\ht\pandoc@box+\dp\pandoc@box\relax}%
      \Gscale@div\@tempb\linewidth{\wd\pandoc@box}%
      % select the smaller of both
      \ifdim\@tempb\p@<\@tempa\p@
        \let\@tempa\@tempb
      \fi
      % scaling accordingly (\@tempa < 1)
      \ifdim\@tempa\p@<\p@
        \scalebox{\@tempa}{\usebox\pandoc@box}%
      % scaling not needed, use as it is
      \else
        \usebox{\pandoc@box}%
      \fi
    }
    \makeatother

    % Define a nice break command that doesn't care if a line doesn't already
    % exist.
    \def\br{\hspace*{\fill} \\* }
    % Math Jax compatibility definitions
    \def\gt{>}
    \def\lt{<}
    \let\Oldtex\TeX
    \let\Oldlatex\LaTeX
    \renewcommand{\TeX}{\textrm{\Oldtex}}
    \renewcommand{\LaTeX}{\textrm{\Oldlatex}}
    % Document parameters
    % Document title
    \title{Single-Q-calibration}
    
    
    
    
    
    
    
% Pygments definitions
\makeatletter
\def\PY@reset{\let\PY@it=\relax \let\PY@bf=\relax%
    \let\PY@ul=\relax \let\PY@tc=\relax%
    \let\PY@bc=\relax \let\PY@ff=\relax}
\def\PY@tok#1{\csname PY@tok@#1\endcsname}
\def\PY@toks#1+{\ifx\relax#1\empty\else%
    \PY@tok{#1}\expandafter\PY@toks\fi}
\def\PY@do#1{\PY@bc{\PY@tc{\PY@ul{%
    \PY@it{\PY@bf{\PY@ff{#1}}}}}}}
\def\PY#1#2{\PY@reset\PY@toks#1+\relax+\PY@do{#2}}

\@namedef{PY@tok@w}{\def\PY@tc##1{\textcolor[rgb]{0.73,0.73,0.73}{##1}}}
\@namedef{PY@tok@c}{\let\PY@it=\textit\def\PY@tc##1{\textcolor[rgb]{0.24,0.48,0.48}{##1}}}
\@namedef{PY@tok@cp}{\def\PY@tc##1{\textcolor[rgb]{0.61,0.40,0.00}{##1}}}
\@namedef{PY@tok@k}{\let\PY@bf=\textbf\def\PY@tc##1{\textcolor[rgb]{0.00,0.50,0.00}{##1}}}
\@namedef{PY@tok@kp}{\def\PY@tc##1{\textcolor[rgb]{0.00,0.50,0.00}{##1}}}
\@namedef{PY@tok@kt}{\def\PY@tc##1{\textcolor[rgb]{0.69,0.00,0.25}{##1}}}
\@namedef{PY@tok@o}{\def\PY@tc##1{\textcolor[rgb]{0.40,0.40,0.40}{##1}}}
\@namedef{PY@tok@ow}{\let\PY@bf=\textbf\def\PY@tc##1{\textcolor[rgb]{0.67,0.13,1.00}{##1}}}
\@namedef{PY@tok@nb}{\def\PY@tc##1{\textcolor[rgb]{0.00,0.50,0.00}{##1}}}
\@namedef{PY@tok@nf}{\def\PY@tc##1{\textcolor[rgb]{0.00,0.00,1.00}{##1}}}
\@namedef{PY@tok@nc}{\let\PY@bf=\textbf\def\PY@tc##1{\textcolor[rgb]{0.00,0.00,1.00}{##1}}}
\@namedef{PY@tok@nn}{\let\PY@bf=\textbf\def\PY@tc##1{\textcolor[rgb]{0.00,0.00,1.00}{##1}}}
\@namedef{PY@tok@ne}{\let\PY@bf=\textbf\def\PY@tc##1{\textcolor[rgb]{0.80,0.25,0.22}{##1}}}
\@namedef{PY@tok@nv}{\def\PY@tc##1{\textcolor[rgb]{0.10,0.09,0.49}{##1}}}
\@namedef{PY@tok@no}{\def\PY@tc##1{\textcolor[rgb]{0.53,0.00,0.00}{##1}}}
\@namedef{PY@tok@nl}{\def\PY@tc##1{\textcolor[rgb]{0.46,0.46,0.00}{##1}}}
\@namedef{PY@tok@ni}{\let\PY@bf=\textbf\def\PY@tc##1{\textcolor[rgb]{0.44,0.44,0.44}{##1}}}
\@namedef{PY@tok@na}{\def\PY@tc##1{\textcolor[rgb]{0.41,0.47,0.13}{##1}}}
\@namedef{PY@tok@nt}{\let\PY@bf=\textbf\def\PY@tc##1{\textcolor[rgb]{0.00,0.50,0.00}{##1}}}
\@namedef{PY@tok@nd}{\def\PY@tc##1{\textcolor[rgb]{0.67,0.13,1.00}{##1}}}
\@namedef{PY@tok@s}{\def\PY@tc##1{\textcolor[rgb]{0.73,0.13,0.13}{##1}}}
\@namedef{PY@tok@sd}{\let\PY@it=\textit\def\PY@tc##1{\textcolor[rgb]{0.73,0.13,0.13}{##1}}}
\@namedef{PY@tok@si}{\let\PY@bf=\textbf\def\PY@tc##1{\textcolor[rgb]{0.64,0.35,0.47}{##1}}}
\@namedef{PY@tok@se}{\let\PY@bf=\textbf\def\PY@tc##1{\textcolor[rgb]{0.67,0.36,0.12}{##1}}}
\@namedef{PY@tok@sr}{\def\PY@tc##1{\textcolor[rgb]{0.64,0.35,0.47}{##1}}}
\@namedef{PY@tok@ss}{\def\PY@tc##1{\textcolor[rgb]{0.10,0.09,0.49}{##1}}}
\@namedef{PY@tok@sx}{\def\PY@tc##1{\textcolor[rgb]{0.00,0.50,0.00}{##1}}}
\@namedef{PY@tok@m}{\def\PY@tc##1{\textcolor[rgb]{0.40,0.40,0.40}{##1}}}
\@namedef{PY@tok@gh}{\let\PY@bf=\textbf\def\PY@tc##1{\textcolor[rgb]{0.00,0.00,0.50}{##1}}}
\@namedef{PY@tok@gu}{\let\PY@bf=\textbf\def\PY@tc##1{\textcolor[rgb]{0.50,0.00,0.50}{##1}}}
\@namedef{PY@tok@gd}{\def\PY@tc##1{\textcolor[rgb]{0.63,0.00,0.00}{##1}}}
\@namedef{PY@tok@gi}{\def\PY@tc##1{\textcolor[rgb]{0.00,0.52,0.00}{##1}}}
\@namedef{PY@tok@gr}{\def\PY@tc##1{\textcolor[rgb]{0.89,0.00,0.00}{##1}}}
\@namedef{PY@tok@ge}{\let\PY@it=\textit}
\@namedef{PY@tok@gs}{\let\PY@bf=\textbf}
\@namedef{PY@tok@ges}{\let\PY@bf=\textbf\let\PY@it=\textit}
\@namedef{PY@tok@gp}{\let\PY@bf=\textbf\def\PY@tc##1{\textcolor[rgb]{0.00,0.00,0.50}{##1}}}
\@namedef{PY@tok@go}{\def\PY@tc##1{\textcolor[rgb]{0.44,0.44,0.44}{##1}}}
\@namedef{PY@tok@gt}{\def\PY@tc##1{\textcolor[rgb]{0.00,0.27,0.87}{##1}}}
\@namedef{PY@tok@err}{\def\PY@bc##1{{\setlength{\fboxsep}{\string -\fboxrule}\fcolorbox[rgb]{1.00,0.00,0.00}{1,1,1}{\strut ##1}}}}
\@namedef{PY@tok@kc}{\let\PY@bf=\textbf\def\PY@tc##1{\textcolor[rgb]{0.00,0.50,0.00}{##1}}}
\@namedef{PY@tok@kd}{\let\PY@bf=\textbf\def\PY@tc##1{\textcolor[rgb]{0.00,0.50,0.00}{##1}}}
\@namedef{PY@tok@kn}{\let\PY@bf=\textbf\def\PY@tc##1{\textcolor[rgb]{0.00,0.50,0.00}{##1}}}
\@namedef{PY@tok@kr}{\let\PY@bf=\textbf\def\PY@tc##1{\textcolor[rgb]{0.00,0.50,0.00}{##1}}}
\@namedef{PY@tok@bp}{\def\PY@tc##1{\textcolor[rgb]{0.00,0.50,0.00}{##1}}}
\@namedef{PY@tok@fm}{\def\PY@tc##1{\textcolor[rgb]{0.00,0.00,1.00}{##1}}}
\@namedef{PY@tok@vc}{\def\PY@tc##1{\textcolor[rgb]{0.10,0.09,0.49}{##1}}}
\@namedef{PY@tok@vg}{\def\PY@tc##1{\textcolor[rgb]{0.10,0.09,0.49}{##1}}}
\@namedef{PY@tok@vi}{\def\PY@tc##1{\textcolor[rgb]{0.10,0.09,0.49}{##1}}}
\@namedef{PY@tok@vm}{\def\PY@tc##1{\textcolor[rgb]{0.10,0.09,0.49}{##1}}}
\@namedef{PY@tok@sa}{\def\PY@tc##1{\textcolor[rgb]{0.73,0.13,0.13}{##1}}}
\@namedef{PY@tok@sb}{\def\PY@tc##1{\textcolor[rgb]{0.73,0.13,0.13}{##1}}}
\@namedef{PY@tok@sc}{\def\PY@tc##1{\textcolor[rgb]{0.73,0.13,0.13}{##1}}}
\@namedef{PY@tok@dl}{\def\PY@tc##1{\textcolor[rgb]{0.73,0.13,0.13}{##1}}}
\@namedef{PY@tok@s2}{\def\PY@tc##1{\textcolor[rgb]{0.73,0.13,0.13}{##1}}}
\@namedef{PY@tok@sh}{\def\PY@tc##1{\textcolor[rgb]{0.73,0.13,0.13}{##1}}}
\@namedef{PY@tok@s1}{\def\PY@tc##1{\textcolor[rgb]{0.73,0.13,0.13}{##1}}}
\@namedef{PY@tok@mb}{\def\PY@tc##1{\textcolor[rgb]{0.40,0.40,0.40}{##1}}}
\@namedef{PY@tok@mf}{\def\PY@tc##1{\textcolor[rgb]{0.40,0.40,0.40}{##1}}}
\@namedef{PY@tok@mh}{\def\PY@tc##1{\textcolor[rgb]{0.40,0.40,0.40}{##1}}}
\@namedef{PY@tok@mi}{\def\PY@tc##1{\textcolor[rgb]{0.40,0.40,0.40}{##1}}}
\@namedef{PY@tok@il}{\def\PY@tc##1{\textcolor[rgb]{0.40,0.40,0.40}{##1}}}
\@namedef{PY@tok@mo}{\def\PY@tc##1{\textcolor[rgb]{0.40,0.40,0.40}{##1}}}
\@namedef{PY@tok@ch}{\let\PY@it=\textit\def\PY@tc##1{\textcolor[rgb]{0.24,0.48,0.48}{##1}}}
\@namedef{PY@tok@cm}{\let\PY@it=\textit\def\PY@tc##1{\textcolor[rgb]{0.24,0.48,0.48}{##1}}}
\@namedef{PY@tok@cpf}{\let\PY@it=\textit\def\PY@tc##1{\textcolor[rgb]{0.24,0.48,0.48}{##1}}}
\@namedef{PY@tok@c1}{\let\PY@it=\textit\def\PY@tc##1{\textcolor[rgb]{0.24,0.48,0.48}{##1}}}
\@namedef{PY@tok@cs}{\let\PY@it=\textit\def\PY@tc##1{\textcolor[rgb]{0.24,0.48,0.48}{##1}}}

\def\PYZbs{\char`\\}
\def\PYZus{\char`\_}
\def\PYZob{\char`\{}
\def\PYZcb{\char`\}}
\def\PYZca{\char`\^}
\def\PYZam{\char`\&}
\def\PYZlt{\char`\<}
\def\PYZgt{\char`\>}
\def\PYZsh{\char`\#}
\def\PYZpc{\char`\%}
\def\PYZdl{\char`\$}
\def\PYZhy{\char`\-}
\def\PYZsq{\char`\'}
\def\PYZdq{\char`\"}
\def\PYZti{\char`\~}
% for compatibility with earlier versions
\def\PYZat{@}
\def\PYZlb{[}
\def\PYZrb{]}
\makeatother


    % For linebreaks inside Verbatim environment from package fancyvrb.
    \makeatletter
        \newbox\Wrappedcontinuationbox
        \newbox\Wrappedvisiblespacebox
        \newcommand*\Wrappedvisiblespace {\textcolor{red}{\textvisiblespace}}
        \newcommand*\Wrappedcontinuationsymbol {\textcolor{red}{\llap{\tiny$\m@th\hookrightarrow$}}}
        \newcommand*\Wrappedcontinuationindent {3ex }
        \newcommand*\Wrappedafterbreak {\kern\Wrappedcontinuationindent\copy\Wrappedcontinuationbox}
        % Take advantage of the already applied Pygments mark-up to insert
        % potential linebreaks for TeX processing.
        %        {, <, #, %, $, ' and ": go to next line.
        %        _, }, ^, &, >, - and ~: stay at end of broken line.
        % Use of \textquotesingle for straight quote.
        \newcommand*\Wrappedbreaksatspecials {%
            \def\PYGZus{\discretionary{\char`\_}{\Wrappedafterbreak}{\char`\_}}%
            \def\PYGZob{\discretionary{}{\Wrappedafterbreak\char`\{}{\char`\{}}%
            \def\PYGZcb{\discretionary{\char`\}}{\Wrappedafterbreak}{\char`\}}}%
            \def\PYGZca{\discretionary{\char`\^}{\Wrappedafterbreak}{\char`\^}}%
            \def\PYGZam{\discretionary{\char`\&}{\Wrappedafterbreak}{\char`\&}}%
            \def\PYGZlt{\discretionary{}{\Wrappedafterbreak\char`\<}{\char`\<}}%
            \def\PYGZgt{\discretionary{\char`\>}{\Wrappedafterbreak}{\char`\>}}%
            \def\PYGZsh{\discretionary{}{\Wrappedafterbreak\char`\#}{\char`\#}}%
            \def\PYGZpc{\discretionary{}{\Wrappedafterbreak\char`\%}{\char`\%}}%
            \def\PYGZdl{\discretionary{}{\Wrappedafterbreak\char`\$}{\char`\$}}%
            \def\PYGZhy{\discretionary{\char`\-}{\Wrappedafterbreak}{\char`\-}}%
            \def\PYGZsq{\discretionary{}{\Wrappedafterbreak\textquotesingle}{\textquotesingle}}%
            \def\PYGZdq{\discretionary{}{\Wrappedafterbreak\char`\"}{\char`\"}}%
            \def\PYGZti{\discretionary{\char`\~}{\Wrappedafterbreak}{\char`\~}}%
        }
        % Some characters . , ; ? ! / are not pygmentized.
        % This macro makes them "active" and they will insert potential linebreaks
        \newcommand*\Wrappedbreaksatpunct {%
            \lccode`\~`\.\lowercase{\def~}{\discretionary{\hbox{\char`\.}}{\Wrappedafterbreak}{\hbox{\char`\.}}}%
            \lccode`\~`\,\lowercase{\def~}{\discretionary{\hbox{\char`\,}}{\Wrappedafterbreak}{\hbox{\char`\,}}}%
            \lccode`\~`\;\lowercase{\def~}{\discretionary{\hbox{\char`\;}}{\Wrappedafterbreak}{\hbox{\char`\;}}}%
            \lccode`\~`\:\lowercase{\def~}{\discretionary{\hbox{\char`\:}}{\Wrappedafterbreak}{\hbox{\char`\:}}}%
            \lccode`\~`\?\lowercase{\def~}{\discretionary{\hbox{\char`\?}}{\Wrappedafterbreak}{\hbox{\char`\?}}}%
            \lccode`\~`\!\lowercase{\def~}{\discretionary{\hbox{\char`\!}}{\Wrappedafterbreak}{\hbox{\char`\!}}}%
            \lccode`\~`\/\lowercase{\def~}{\discretionary{\hbox{\char`\/}}{\Wrappedafterbreak}{\hbox{\char`\/}}}%
            \catcode`\.\active
            \catcode`\,\active
            \catcode`\;\active
            \catcode`\:\active
            \catcode`\?\active
            \catcode`\!\active
            \catcode`\/\active
            \lccode`\~`\~
        }
    \makeatother

    \let\OriginalVerbatim=\Verbatim
    \makeatletter
    \renewcommand{\Verbatim}[1][1]{%
        %\parskip\z@skip
        \sbox\Wrappedcontinuationbox {\Wrappedcontinuationsymbol}%
        \sbox\Wrappedvisiblespacebox {\FV@SetupFont\Wrappedvisiblespace}%
        \def\FancyVerbFormatLine ##1{\hsize\linewidth
            \vtop{\raggedright\hyphenpenalty\z@\exhyphenpenalty\z@
                \doublehyphendemerits\z@\finalhyphendemerits\z@
                \strut ##1\strut}%
        }%
        % If the linebreak is at a space, the latter will be displayed as visible
        % space at end of first line, and a continuation symbol starts next line.
        % Stretch/shrink are however usually zero for typewriter font.
        \def\FV@Space {%
            \nobreak\hskip\z@ plus\fontdimen3\font minus\fontdimen4\font
            \discretionary{\copy\Wrappedvisiblespacebox}{\Wrappedafterbreak}
            {\kern\fontdimen2\font}%
        }%

        % Allow breaks at special characters using \PYG... macros.
        \Wrappedbreaksatspecials
        % Breaks at punctuation characters . , ; ? ! and / need catcode=\active
        \OriginalVerbatim[#1,codes*=\Wrappedbreaksatpunct]%
    }
    \makeatother

    % Exact colors from NB
    \definecolor{incolor}{HTML}{303F9F}
    \definecolor{outcolor}{HTML}{D84315}
    \definecolor{cellborder}{HTML}{CFCFCF}
    \definecolor{cellbackground}{HTML}{F7F7F7}

    % prompt
    \makeatletter
    \newcommand{\boxspacing}{\kern\kvtcb@left@rule\kern\kvtcb@boxsep}
    \makeatother
    \newcommand{\prompt}[4]{
        {\ttfamily\llap{{\color{#2}[#3]:\hspace{3pt}#4}}\vspace{-\baselineskip}}
    }
    

    
    % Prevent overflowing lines due to hard-to-break entities
    \sloppy
    % Setup hyperref package
    \hypersetup{
      breaklinks=true,  % so long urls are correctly broken across lines
      colorlinks=true,
      urlcolor=urlcolor,
      linkcolor=linkcolor,
      citecolor=citecolor,
      }
    % Slightly bigger margins than the latex defaults
    
    \geometry{verbose,tmargin=1in,bmargin=1in,lmargin=1in,rmargin=1in}
    
    

\begin{document}
    
    \maketitle
    
    

    
    \section{前言}\label{ux524dux8a00}

    这是一个着重于描述单比特操作和读取校准逻辑的文档。有关如何定义波形,如何自定义门的内容请阅读\texttt{qlisp}的相关文档。校准中使用的实验代码示例为\texttt{Scanner2},其相关定义方式不在此赘述。使用任意实验方式均可,但校准逻辑不会有太大的变化。

另外,这是一个可以运行的\texttt{Notebook}但不建议在这个里面运行,这是出于对执行代码简洁性的考虑。

    \section{单比特校准的理论基础}\label{ux5355ux6bd4ux7279ux6821ux51c6ux7684ux7406ux8bbaux57faux7840}

    \subsection{驱动和操作过程}\label{ux9a71ux52a8ux548cux64cdux4f5cux8fc7ux7a0b}

    按照薛定谔方程的描述,量子态\(|\psi(t)\rangle\)的演化可以用演化算符\(U(t)\)来描述。

\[
i\frac{\partial}{\partial t}|\psi(t)\rangle=H|\psi(t)\rangle\\
|\psi'\rangle=U|\psi\rangle\\\notag
U(t)={\mathcal T}\exp\left[-i\int_0^tH(t'){\rm d}t'\right]
\]

实现任意操作的核心,在于构建适当的演化算符。在一个 two-level system
(TLS) 中,任意的哈密顿量可以写为,

\[
H\equiv\begin{pmatrix}a+z&x-iy\\x+iy&a-z\end{pmatrix}=aI+xX+yY+zZ\\
I=\begin{pmatrix}1&\\&1\end{pmatrix}, X=\begin{pmatrix}&1\\1&\end{pmatrix}, Y=\begin{pmatrix}&-i\\i&\end{pmatrix}, Z=\begin{pmatrix}1&\\&-1\end{pmatrix}
\]

这等效于 Bloch 球上的旋转操作。Bloch球的定义见下文。

在我们的 transmon qubit
中,一般考虑其基态\(|0\rangle\)和第一激发态\(|1\rangle\),矩阵表示为 \[
|0\rangle\equiv\begin{pmatrix}1\\0\end{pmatrix}, |1\rangle\equiv\begin{pmatrix}0\\1\end{pmatrix}
\]

对应能量\(E_0\)和\(E_1\),则其哈密顿量可以写为

\[
H_0=E_0|0\rangle\langle0|+E_1|1\rangle\langle1|=\begin{pmatrix}E_0&\\&E_1\end{pmatrix}=\begin{pmatrix}\frac{E_0+E_1}2-\frac{E_1-E_0}2&\\&\frac{E_0+E_1}2+\frac{E_1-E_0}2\end{pmatrix}=-\frac\omega2Z+\frac{E_0+E_1}2I
\]

其中\(\omega=E_1-E_0\),且忽略常数项后,得到

\[
H_0=-\frac\omega2Z
\]

若进入相互作用表象考虑,选定表象变换算符\(\tilde U=e^{-iH_0t}\),

\[
|\psi_I\rangle=\tilde U|\psi_S\rangle\\
H_I=\tilde U^\dagger H_S\tilde U+i\frac{\partial \tilde U^\dagger}{\partial t}\tilde U=e^{iH_0t}H_0e^{-iH_0t}+i\left(\frac{\partial }{\partial t}e^{iH_0t}\right)e^{-iH_0t}=H_0+i\cdot iH_0\cdot e^{iH_0t}e^{-iH_0t}=H_0-H_0=0
\]

即,在相互作用表象下固有哈密顿量\(H_0\)的影响可以先不考虑。

\begin{quote}
这里有一个在表述上留下的``坑'',请格外注意。
\end{quote}

实际操作往往通过给比特一个微波驱动来实现,其哈密顿量为

\[
\begin{aligned}
H'=\Omega e^{i(\omega_dt-\phi)}|0\rangle\langle1|+h.c.
&=\Omega\left(\left(\cos{(\omega_dt-\phi)}+i\sin{(\omega_dt-\phi)}\right)|0\rangle\langle1|+\left(\cos{(\omega_dt-\phi)}-i\sin{(\omega_dt-\phi)}\right)|1\rangle\langle0|\right)\\
&=\Omega\left(X\cos{(\omega_dt-\phi)}-Y\sin{(\omega_dt-\phi)}\right)
\end{aligned}
\]

此时,系统哈密顿量为\(H=H_0+H'\),而在相互作用表象下,满足共振驱动条件\(\omega_d=\omega\),且考虑旋波近似
(RWA),

\[
H_I=\tilde U^\dagger H'\tilde U=\Omega\left(X\cos\phi+Y\sin\phi\right)
\]

那么此时的系统演化算符\({\rm rfUnitary}(\theta, \phi)\),

\[
{\rm rfUnitary}(\theta, \phi)=e^{-i\frac\theta2\left(X\cos\phi+Y\sin\phi\right)}
\]

在这个演化过程中,\(XY\)-平面上的旋转轴\(\phi\)通过微波驱动的初相位来调控。而旋转的角度\(\theta\)则是需要校准的量,可以看出\(\theta\propto \Omega t\),\(\Omega\)是耦合强度,也就是驱动微波的振幅,而\(t\)就是驱动微波时间,即驱动微波的包络下面积决定转动的角度。

一般通过标定\(\theta=\pi/2\),得到一组演化算符

\[
{\rm R}(\phi)={\rm rfUnitary}(\pi/2, \phi)
\]

加上任意的相位操作,即\({\rm P}(\phi)=|0\rangle\langle0|+e^{-i\phi}|1\rangle\langle1|\),可以组成任意的单比特上的酉操作。

    \subsection{读取过程}\label{ux8bfbux53d6ux8fc7ux7a0b}

    transmon在系统中和一个谐振腔通过电容耦合,耦合强度与频率失谐量近似满足色散耦合的条件,此时可以将其理解为:比特处在不同的态上,谐振腔的频率不同。通过探测这个频率变化带来的响应,可以实现量子态的区分。同时考虑到耦合形式,这个读取是量子非破坏性的
(QND) 的。

一般情况下,我们把一个频率为\(\omega_r\)的波包给到读取用谐振腔上,采集其透射或者反射的时域信号,记为\(\rm Sig\),对应信号为\texttt{\textquotesingle{}trace\_avg\textquotesingle{}}。

对此时域信号\(\rm Sig\)进行给定频率\(\omega_r\)的数字解模,得到其复IQ平面内的复振幅,记录为\(S_{21}\),此读取方式记录为\texttt{iq},按照测量次数对多个
shots
中的\(S_{21}\)作平均,得到的信号格式为\texttt{\textquotesingle{}iq\_avg\textquotesingle{}}。

在同一个读取频率下,比特在\(|0\rangle\)和在\(|1\rangle\)时得到的\(S_{21}\)位于IQ平面的不同位置,考虑到噪声涨落,其表现为以不同点为中心的二维高斯分布。在IQ平面内进行区域判别,可以判断比特处于\(|0\rangle\)或者\(|1\rangle\),此信号为\texttt{\textquotesingle{}state\textquotesingle{}},即每一次shot都统计所有\texttt{cbit}对应的联合结果。对不同shots的态的测量进行统计,得到信号为\texttt{\textquotesingle{}count\textquotesingle{}}。若不关心不同的\texttt{\textquotesingle{}cbit\textquotesingle{}}的关联,可以进行单比特的概率计算,此信号为\texttt{\textquotesingle{}population\textquotesingle{}},靠近1表示比特在\(|1\rangle\)的概率。

在部分设备下面,支持在采集AD端初步处理完数据再回传,所以这一类信号可以在对应的\texttt{signal}前加\texttt{\textquotesingle{}remote\_\textquotesingle{}}前缀。

    \section{校准策略}\label{ux6821ux51c6ux7b56ux7565}

    我们的校准策略以拿到一块新样品为例,即,完全不知道样品的任何信息。这里的目的是校准出能用的单比特门,对于比特性质的表征,如\(T_1\)和\(T_2\)的测量不展开叙述。一些优化和继续提高的方法,特别是读取的优化,将会放在单比特进阶的文档中。

一般情况按照以下步骤进行: 1.
利用网分NA/采集卡AD做\texttt{S21}实验,找到比特\(|0\rangle\)时对应的腔频,将读取频率设置在此频率,跳转2.
2.
在\texttt{\textquotesingle{}iq\_avg\textquotesingle{}}或者\texttt{\textquotesingle{}population\textquotesingle{}}信号下做比特的01\texttt{Spectrum}实验,得到可能得01驱动频率,跳转3.
3.
在\texttt{\textquotesingle{}iq\_avg\textquotesingle{}}或者\texttt{\textquotesingle{}population\textquotesingle{}}信号下做比特的01\texttt{Ramsey}实验,确认频率是否正确,一般情况跳转4,否则回到2.
4.
在\texttt{\textquotesingle{}iq\_avg\textquotesingle{}}或者\texttt{\textquotesingle{}population\textquotesingle{}}信号下做比特的01\texttt{PowerRabi}实验,若未实现
single shot 则跳转5,若为了校准drive则跳转6 5.
做01态判别的\texttt{Scatter}实验,两态可区分后\texttt{\textquotesingle{}population\textquotesingle{}}信号可用,若需要继续校准则一般跳转3.
6.
在\texttt{\textquotesingle{}population\textquotesingle{}}信号下做比特的01\texttt{Delta}实验,若找到则需要和步骤4来回迭代,直到达到理想精度

确认保真度则一般通过两个实验来实现 7.
\texttt{\textquotesingle{}Count\textquotesingle{}}实验 8.
\texttt{\textquotesingle{}RB\textquotesingle{}}实验

    \begin{tcolorbox}[breakable, size=fbox, boxrule=1pt, pad at break*=1mm,colback=cellbackground, colframe=cellborder]
\prompt{In}{incolor}{8}{\boxspacing}
\begin{Verbatim}[commandchars=\\\{\}]
\PY{o}{\PYZpc{}}\PY{k}{matplotlib} notebook

\PY{k+kn}{import}\PY{+w}{ }\PY{n+nn}{numpy}\PY{+w}{ }\PY{k}{as}\PY{+w}{ }\PY{n+nn}{np}
\PY{k+kn}{from}\PY{+w}{ }\PY{n+nn}{matplotlib}\PY{+w}{ }\PY{k+kn}{import} \PY{n}{pyplot} \PY{k}{as} \PY{n}{plt}

\PY{k+kn}{import}\PY{+w}{ }\PY{n+nn}{kernel}
\PY{c+c1}{\PYZsh{} kernel.init()}
\PY{k+kn}{from}\PY{+w}{ }\PY{n+nn}{qos\PYZus{}tools}\PY{n+nn}{.}\PY{n+nn}{experiment}\PY{n+nn}{.}\PY{n+nn}{scanner2}\PY{+w}{ }\PY{k+kn}{import} \PY{n}{Scanner}

\PY{k+kn}{from}\PY{+w}{ }\PY{n+nn}{itertools}\PY{+w}{ }\PY{k+kn}{import} \PY{n}{chain}
\PY{k+kn}{from}\PY{+w}{ }\PY{n+nn}{typing}\PY{+w}{ }\PY{k+kn}{import} \PY{n}{Optional}\PY{p}{,} \PY{n}{Any}\PY{p}{,} \PY{n}{Union}
\PY{k+kn}{from}\PY{+w}{ }\PY{n+nn}{qos\PYZus{}tools}\PY{n+nn}{.}\PY{n+nn}{experiment}\PY{n+nn}{.}\PY{n+nn}{libs}\PY{n+nn}{.}\PY{n+nn}{tools}\PY{+w}{ }\PY{k+kn}{import} \PY{n}{generate\PYZus{}spanlist}
\PY{k+kn}{from}\PY{+w}{ }\PY{n+nn}{home}\PY{n+nn}{.}\PY{n+nn}{hkxu}\PY{n+nn}{.}\PY{n+nn}{tools}\PY{+w}{ }\PY{k+kn}{import} \PY{n}{get\PYZus{}record\PYZus{}by\PYZus{}id}
\PY{k+kn}{from}\PY{+w}{ }\PY{n+nn}{waveforms}\PY{n+nn}{.}\PY{n+nn}{visualization}\PY{n+nn}{.}\PY{n+nn}{widgets}\PY{+w}{ }\PY{k+kn}{import} \PY{n}{DataPicker}

\PY{n}{plt}\PY{o}{.}\PY{n}{rcParams}\PY{p}{[}\PY{l+s+s1}{\PYZsq{}}\PY{l+s+s1}{xtick.direction}\PY{l+s+s1}{\PYZsq{}}\PY{p}{]} \PY{o}{=} \PY{l+s+s1}{\PYZsq{}}\PY{l+s+s1}{in}\PY{l+s+s1}{\PYZsq{}}
\PY{n}{plt}\PY{o}{.}\PY{n}{rcParams}\PY{p}{[}\PY{l+s+s1}{\PYZsq{}}\PY{l+s+s1}{ytick.direction}\PY{l+s+s1}{\PYZsq{}}\PY{p}{]} \PY{o}{=} \PY{l+s+s1}{\PYZsq{}}\PY{l+s+s1}{in}\PY{l+s+s1}{\PYZsq{}}
\PY{n}{plt}\PY{o}{.}\PY{n}{rcParams}\PY{p}{[}\PY{l+s+s1}{\PYZsq{}}\PY{l+s+s1}{xtick.top}\PY{l+s+s1}{\PYZsq{}}\PY{p}{]} \PY{o}{=} \PY{k+kc}{True}
\PY{n}{plt}\PY{o}{.}\PY{n}{rcParams}\PY{p}{[}\PY{l+s+s1}{\PYZsq{}}\PY{l+s+s1}{ytick.right}\PY{l+s+s1}{\PYZsq{}}\PY{p}{]} \PY{o}{=} \PY{k+kc}{True}
\PY{n}{plt}\PY{o}{.}\PY{n}{rcParams}\PY{p}{[}\PY{l+s+s1}{\PYZsq{}}\PY{l+s+s1}{xtick.minor.visible}\PY{l+s+s1}{\PYZsq{}}\PY{p}{]} \PY{o}{=} \PY{k+kc}{True}
\PY{n}{plt}\PY{o}{.}\PY{n}{rcParams}\PY{p}{[}\PY{l+s+s1}{\PYZsq{}}\PY{l+s+s1}{ytick.minor.visible}\PY{l+s+s1}{\PYZsq{}}\PY{p}{]} \PY{o}{=} \PY{k+kc}{True}
\PY{n}{plt}\PY{o}{.}\PY{n}{rcParams}\PY{p}{[}\PY{l+s+s1}{\PYZsq{}}\PY{l+s+s1}{image.origin}\PY{l+s+s1}{\PYZsq{}}\PY{p}{]} \PY{o}{=} \PY{l+s+s1}{\PYZsq{}}\PY{l+s+s1}{lower}\PY{l+s+s1}{\PYZsq{}}
\PY{n}{plt}\PY{o}{.}\PY{n}{rcParams}\PY{p}{[}\PY{l+s+s1}{\PYZsq{}}\PY{l+s+s1}{figure.figsize}\PY{l+s+s1}{\PYZsq{}}\PY{p}{]} \PY{o}{=} \PY{p}{[}\PY{l+m+mi}{9}\PY{p}{,} \PY{l+m+mi}{3}\PY{p}{]}
\PY{n}{plt}\PY{o}{.}\PY{n}{rcParams}\PY{p}{[}\PY{l+s+s1}{\PYZsq{}}\PY{l+s+s1}{font.size}\PY{l+s+s1}{\PYZsq{}}\PY{p}{]} \PY{o}{=} \PY{l+m+mi}{8}
\PY{n}{plt}\PY{o}{.}\PY{n}{rcParams}\PY{p}{[}\PY{l+s+s1}{\PYZsq{}}\PY{l+s+s1}{lines.linewidth}\PY{l+s+s1}{\PYZsq{}}\PY{p}{]} \PY{o}{=} \PY{l+m+mi}{1}
\PY{n}{plt}\PY{o}{.}\PY{n}{rcParams}\PY{p}{[}\PY{l+s+s1}{\PYZsq{}}\PY{l+s+s1}{lines.markersize}\PY{l+s+s1}{\PYZsq{}}\PY{p}{]} \PY{o}{=} \PY{l+m+mi}{2}
\PY{n}{plt}\PY{o}{.}\PY{n}{rcParams}\PY{p}{[}\PY{l+s+s1}{\PYZsq{}}\PY{l+s+s1}{lines.marker}\PY{l+s+s1}{\PYZsq{}}\PY{p}{]} \PY{o}{=} \PY{l+s+s1}{\PYZsq{}}\PY{l+s+s1}{.}\PY{l+s+s1}{\PYZsq{}}
\PY{n}{plt}\PY{o}{.}\PY{n}{rcParams}\PY{p}{[}\PY{l+s+s1}{\PYZsq{}}\PY{l+s+s1}{pdf.fonttype}\PY{l+s+s1}{\PYZsq{}}\PY{p}{]} \PY{o}{=} \PY{l+m+mi}{42}
\PY{n}{plt}\PY{o}{.}\PY{n}{rcParams}\PY{p}{[}\PY{l+s+s1}{\PYZsq{}}\PY{l+s+s1}{ps.fonttype}\PY{l+s+s1}{\PYZsq{}}\PY{p}{]} \PY{o}{=} \PY{l+m+mi}{42}
\PY{n}{plt}\PY{o}{.}\PY{n}{rcParams}\PY{p}{[}\PY{l+s+s1}{\PYZsq{}}\PY{l+s+s1}{xtick.labelsize}\PY{l+s+s1}{\PYZsq{}}\PY{p}{]} \PY{o}{=} \PY{l+m+mi}{6}
\PY{n}{plt}\PY{o}{.}\PY{n}{rcParams}\PY{p}{[}\PY{l+s+s1}{\PYZsq{}}\PY{l+s+s1}{ytick.labelsize}\PY{l+s+s1}{\PYZsq{}}\PY{p}{]} \PY{o}{=} \PY{l+m+mi}{6}
\end{Verbatim}
\end{tcolorbox}

    \begin{tcolorbox}[breakable, size=fbox, boxrule=1pt, pad at break*=1mm,colback=cellbackground, colframe=cellborder]
\prompt{In}{incolor}{13}{\boxspacing}
\begin{Verbatim}[commandchars=\\\{\}]
\PY{k+kn}{import}\PY{+w}{ }\PY{n+nn}{time}
\PY{k+kn}{from}\PY{+w}{ }\PY{n+nn}{functools}\PY{+w}{ }\PY{k+kn}{import} \PY{n}{lru\PYZus{}cache}

\PY{n}{default\PYZus{}shots} \PY{o}{=} \PY{n}{kernel}\PY{o}{.}\PY{n}{get}\PY{p}{(}\PY{l+s+s1}{\PYZsq{}}\PY{l+s+s1}{station.shots}\PY{l+s+s1}{\PYZsq{}}\PY{p}{)}
\PY{n+nb}{print}\PY{p}{(}\PY{n}{default\PYZus{}shots}\PY{p}{)}

\PY{n+nd}{@lru\PYZus{}cache}\PY{p}{(}\PY{n}{maxsize}\PY{o}{=}\PY{k+kc}{None}\PY{p}{)}
\PY{k}{def}\PY{+w}{ }\PY{n+nf}{init\PYZus{}bias}\PY{p}{(}\PY{p}{)}\PY{p}{:}
    \PY{n}{ret} \PY{o}{=} \PY{p}{[}\PY{p}{]}
    \PY{k}{return} \PY{n}{ret}

\PY{n+nd}{@lru\PYZus{}cache}\PY{p}{(}\PY{n}{maxsize}\PY{o}{=}\PY{k+kc}{None}\PY{p}{)}
\PY{k}{def}\PY{+w}{ }\PY{n+nf}{fina\PYZus{}bias}\PY{p}{(}\PY{p}{)}\PY{p}{:}
    \PY{n}{ret} \PY{o}{=} \PY{p}{[}\PY{p}{]}
    \PY{k}{return} \PY{n}{ret}

\PY{k}{def}\PY{+w}{ }\PY{n+nf}{general\PYZus{}run\PYZus{}task}\PY{p}{(}\PY{n}{para\PYZus{}dict}\PY{p}{,} \PY{n}{timeout}\PY{p}{,} \PY{n}{print\PYZus{}task}\PY{o}{=}\PY{k+kc}{True}\PY{p}{,} \PY{n}{bar}\PY{o}{=}\PY{k+kc}{True}\PY{p}{,} \PY{n}{default\PYZus{}shots}\PY{o}{=}\PY{n}{default\PYZus{}shots}\PY{p}{,} \PY{o}{*}\PY{o}{*}\PY{n}{kw}\PY{p}{)}\PY{p}{:}
    
    \PY{n}{test} \PY{o}{=} \PY{n}{kernel}\PY{o}{.}\PY{n}{create\PYZus{}task}\PY{p}{(}\PY{n}{Scanner}\PY{p}{,} \PY{n}{args}\PY{o}{=}\PY{p}{(}\PY{p}{)}\PY{p}{,} \PY{n}{kwds}\PY{o}{=}\PY{n}{para\PYZus{}dict}\PY{p}{[}\PY{l+s+s1}{\PYZsq{}}\PY{l+s+s1}{init}\PY{l+s+s1}{\PYZsq{}}\PY{p}{]}\PY{p}{)}
    \PY{n}{test}\PY{o}{.}\PY{n}{init}\PY{p}{(}\PY{o}{*}\PY{o}{*}\PY{n}{para\PYZus{}dict}\PY{p}{)}
    \PY{n}{test}\PY{o}{.}\PY{n}{shots} \PY{o}{=} \PY{n}{default\PYZus{}shots}
    \PY{n}{time}\PY{o}{.}\PY{n}{sleep}\PY{p}{(}\PY{l+m+mf}{0.1}\PY{p}{)}
    \PY{n}{task} \PY{o}{=} \PY{n}{kernel}\PY{o}{.}\PY{n}{submit}\PY{p}{(}\PY{n}{test}\PY{p}{,} \PY{o}{*}\PY{o}{*}\PY{n}{kw}\PY{p}{)}
    \PY{k}{if} \PY{n}{bar}\PY{p}{:}
        \PY{n}{task}\PY{o}{.}\PY{n}{bar}\PY{p}{(}\PY{p}{)}
    \PY{n}{time}\PY{o}{.}\PY{n}{sleep}\PY{p}{(}\PY{l+m+mf}{0.1}\PY{p}{)}
    \PY{n}{task}\PY{o}{.}\PY{n}{join}\PY{p}{(}\PY{n}{timeout}\PY{p}{)}
    \PY{n}{time}\PY{o}{.}\PY{n}{sleep}\PY{p}{(}\PY{l+m+mf}{0.1}\PY{p}{)}
    \PY{k}{if} \PY{n}{print\PYZus{}task}\PY{p}{:}
        \PY{n+nb}{print}\PY{p}{(}\PY{n}{task}\PY{p}{)}
    \PY{k}{return} \PY{n}{task}
\end{Verbatim}
\end{tcolorbox}

    \begin{Verbatim}[commandchars=\\\{\}]
1024
    \end{Verbatim}

    \subsection{Measure}\label{measure}

    \subsubsection{\texorpdfstring{\texttt{S21}}{S21}}\label{s21}

    实验目标:测量

实验原理:

前置条件:无

获取结果

    \begin{tcolorbox}[breakable, size=fbox, boxrule=1pt, pad at break*=1mm,colback=cellbackground, colframe=cellborder]
\prompt{In}{incolor}{26}{\boxspacing}
\begin{Verbatim}[commandchars=\\\{\}]
\PY{k}{def}\PY{+w}{ }\PY{n+nf}{S21}\PY{p}{(}\PY{n}{qubits}\PY{p}{:} \PY{n+nb}{list}\PY{p}{[}\PY{n+nb}{str}\PY{p}{]}\PY{p}{,}
                   \PY{n}{center}\PY{p}{:} \PY{n}{Optional}\PY{p}{[}\PY{n}{Union}\PY{p}{[}\PY{n+nb}{float}\PY{p}{,} \PY{n+nb}{list}\PY{p}{[}\PY{n+nb}{float}\PY{p}{]}\PY{p}{]}\PY{p}{]} \PY{o}{=} \PY{k+kc}{None}\PY{p}{,} \PY{n}{delta}\PY{p}{:} \PY{n}{Optional}\PY{p}{[}\PY{n+nb}{float}\PY{p}{]} \PY{o}{=} \PY{k+kc}{None}\PY{p}{,} \PY{n}{st}\PY{p}{:} \PY{n}{Optional}\PY{p}{[}\PY{n+nb}{float}\PY{p}{]} \PY{o}{=} \PY{k+kc}{None}\PY{p}{,} \PY{n}{ed}\PY{p}{:} \PY{n}{Optional}\PY{p}{[}\PY{n+nb}{float}\PY{p}{]} \PY{o}{=} \PY{k+kc}{None}\PY{p}{,} \PY{n}{mode}\PY{p}{:} \PY{n+nb}{str} \PY{o}{=} \PY{l+s+s1}{\PYZsq{}}\PY{l+s+s1}{linear}\PY{l+s+s1}{\PYZsq{}}\PY{p}{,} \PY{n}{sweep\PYZus{}points}\PY{p}{:} \PY{n+nb}{int} \PY{o}{=} \PY{l+m+mi}{101}\PY{p}{,}
                   \PY{n}{signal}\PY{p}{:} \PY{n+nb}{str} \PY{o}{=} \PY{l+s+s1}{\PYZsq{}}\PY{l+s+s1}{iq\PYZus{}avg}\PY{l+s+s1}{\PYZsq{}}\PY{p}{,} \PY{n}{repeat}\PY{o}{=}\PY{l+m+mi}{1}\PY{p}{,} \PY{o}{*}\PY{o}{*}\PY{n}{kw}\PY{p}{)} \PY{o}{\PYZhy{}}\PY{o}{\PYZgt{}} \PY{n+nb}{dict}\PY{p}{:}
\PY{+w}{    }\PY{l+s+sd}{\PYZdq{}\PYZdq{}\PYZdq{}}
\PY{l+s+sd}{    [f\PYZsq{}Q\PYZob{}i\PYZcb{}\PYZsq{}] Measure S21 without constraints, change awg frequency.}

\PY{l+s+sd}{    Args:}
\PY{l+s+sd}{        qubits (list[str]): qubit names.}
\PY{l+s+sd}{        center (Optional[Union[float, list[float]]], optional): sweep center. Defaults to None.}
\PY{l+s+sd}{        delta (Optional[float], optional): sweep span. Defaults to None.}
\PY{l+s+sd}{        st (Optional[float], optional): sweep start. Defaults to None.}
\PY{l+s+sd}{        ed (Optional[float], optional): sweep end. Defaults to None.}
\PY{l+s+sd}{        sweep\PYZus{}points (int, optional): sweep points. Defaults to 101.}
\PY{l+s+sd}{        mode (str, optional): sweep mode. Defaults to \PYZsq{}linear\PYZsq{}.}
\PY{l+s+sd}{        signal (str, optional): signal. Defaults to \PYZsq{}iq\PYZus{}avg\PYZsq{}.}
\PY{l+s+sd}{    \PYZdq{}\PYZdq{}\PYZdq{}}
    
    \PY{n}{cts} \PY{o}{=} \PY{p}{\PYZob{}}\PY{n}{q}\PY{p}{:} \PY{n}{kernel}\PY{o}{.}\PY{n}{get}\PY{p}{(}\PY{l+s+sa}{f}\PY{l+s+s1}{\PYZsq{}}\PY{l+s+s1}{gate.Measure.}\PY{l+s+si}{\PYZob{}}\PY{n}{q}\PY{l+s+si}{\PYZcb{}}\PY{l+s+s1}{.default\PYZus{}type}\PY{l+s+s1}{\PYZsq{}}\PY{p}{)} \PY{k}{for} \PY{n}{q} \PY{o+ow}{in} \PY{n}{qubits}\PY{p}{\PYZcb{}}
    \PY{n}{cts} \PY{o}{=} \PY{p}{\PYZob{}}\PY{n}{q}\PY{p}{:} \PY{l+s+s1}{\PYZsq{}}\PY{l+s+s1}{params}\PY{l+s+s1}{\PYZsq{}} \PY{k}{if} \PY{n}{cts}\PY{p}{[}\PY{n}{q}\PY{p}{]}\PY{o}{==}\PY{l+s+s1}{\PYZsq{}}\PY{l+s+s1}{default}\PY{l+s+s1}{\PYZsq{}} \PY{k}{else} \PY{n}{cts}\PY{p}{[}\PY{n}{q}\PY{p}{]} \PY{k}{for} \PY{n}{q} \PY{o+ow}{in} \PY{n}{qubits}\PY{p}{\PYZcb{}}

    \PY{k}{if} \PY{n}{center} \PY{o+ow}{is} \PY{k+kc}{None}\PY{p}{:}
        \PY{n}{center} \PY{o}{=} \PY{p}{[}\PY{n}{kernel}\PY{o}{.}\PY{n}{get}\PY{p}{(}
            \PY{l+s+sa}{f}\PY{l+s+s1}{\PYZsq{}}\PY{l+s+s1}{gate.Measure.}\PY{l+s+si}{\PYZob{}}\PY{n}{q}\PY{l+s+si}{\PYZcb{}}\PY{l+s+s1}{.}\PY{l+s+si}{\PYZob{}}\PY{n}{cts}\PY{p}{[}\PY{n}{q}\PY{p}{]}\PY{l+s+si}{\PYZcb{}}\PY{l+s+s1}{.frequency}\PY{l+s+s1}{\PYZsq{}}\PY{p}{)} \PY{k}{for} \PY{n}{q} \PY{o+ow}{in} \PY{n}{qubits}\PY{p}{]}
        
    \PY{k}{elif} \PY{n+nb}{isinstance}\PY{p}{(}\PY{n}{center}\PY{p}{,} \PY{n+nb}{float}\PY{p}{)}\PY{p}{:}
        \PY{n}{center} \PY{o}{=} \PY{p}{[}\PY{n}{center}\PY{p}{]}\PY{o}{*}\PY{n+nb}{len}\PY{p}{(}\PY{n}{qubits}\PY{p}{)}

    \PY{n}{sweep\PYZus{}list} \PY{o}{=} \PY{n}{generate\PYZus{}spanlist}\PY{p}{(}
        \PY{n}{center}\PY{o}{=}\PY{l+m+mi}{0}\PY{p}{,} \PY{n}{delta}\PY{o}{=}\PY{n}{delta}\PY{p}{,} \PY{n}{st}\PY{o}{=}\PY{n}{st}\PY{p}{,} \PY{n}{ed}\PY{o}{=}\PY{n}{ed}\PY{p}{,} \PY{n}{sweep\PYZus{}points}\PY{o}{=}\PY{n}{sweep\PYZus{}points}\PY{p}{,} \PY{n}{mode}\PY{o}{=}\PY{n}{mode}\PY{p}{)}

    \PY{k}{return} \PY{p}{\PYZob{}}
        \PY{l+s+s1}{\PYZsq{}}\PY{l+s+s1}{init}\PY{l+s+s1}{\PYZsq{}}\PY{p}{:} \PY{p}{\PYZob{}}
            \PY{l+s+s1}{\PYZsq{}}\PY{l+s+s1}{name}\PY{l+s+s1}{\PYZsq{}}\PY{p}{:} \PY{l+s+s1}{\PYZsq{}}\PY{l+s+s1}{S21}\PY{l+s+s1}{\PYZsq{}}\PY{p}{,}
            \PY{l+s+s1}{\PYZsq{}}\PY{l+s+s1}{qubits}\PY{l+s+s1}{\PYZsq{}}\PY{p}{:} \PY{n}{qubits}\PY{p}{,}
            \PY{l+s+s1}{\PYZsq{}}\PY{l+s+s1}{signal}\PY{l+s+s1}{\PYZsq{}}\PY{p}{:} \PY{n}{signal}\PY{p}{,}
        \PY{p}{\PYZcb{}}\PY{p}{,}
        \PY{l+s+s1}{\PYZsq{}}\PY{l+s+s1}{setting}\PY{l+s+s1}{\PYZsq{}}\PY{p}{:} \PY{p}{\PYZob{}}
            \PY{l+s+s1}{\PYZsq{}}\PY{l+s+s1}{circuit}\PY{l+s+s1}{\PYZsq{}}\PY{p}{:}
            \PY{n}{init\PYZus{}bias}\PY{p}{(}\PY{p}{)}\PY{o}{+}
            \PY{p}{[}
                \PY{p}{(}\PY{l+s+s1}{\PYZsq{}}\PY{l+s+s1}{Barrier}\PY{l+s+s1}{\PYZsq{}}\PY{p}{,} \PY{n+nb}{tuple}\PY{p}{(}\PY{n}{qubits}\PY{p}{)}\PY{p}{)}\PY{p}{,}
                \PY{p}{(}\PY{p}{(}\PY{l+s+s1}{\PYZsq{}}\PY{l+s+s1}{Delay}\PY{l+s+s1}{\PYZsq{}}\PY{p}{,} \PY{l+m+mf}{2e\PYZhy{}6}\PY{p}{)}\PY{p}{,} \PY{n}{qubits}\PY{p}{[}\PY{l+m+mi}{0}\PY{p}{]}\PY{p}{)}\PY{p}{,}
                \PY{p}{(}\PY{l+s+s1}{\PYZsq{}}\PY{l+s+s1}{Barrier}\PY{l+s+s1}{\PYZsq{}}\PY{p}{,} \PY{n+nb}{tuple}\PY{p}{(}\PY{n}{qubits}\PY{p}{)}\PY{p}{)}\PY{p}{,}
                \PY{o}{*}\PY{p}{[}\PY{p}{(}\PY{p}{(}\PY{l+s+s1}{\PYZsq{}}\PY{l+s+s1}{Measure}\PY{l+s+s1}{\PYZsq{}}\PY{p}{,} \PY{n}{j}\PY{p}{)}\PY{p}{,} \PY{n}{q}\PY{p}{)} \PY{k}{for} \PY{n}{j}\PY{p}{,} \PY{n}{q} \PY{o+ow}{in} \PY{n+nb}{enumerate}\PY{p}{(}\PY{n}{qubits}\PY{p}{)}\PY{p}{]}\PY{p}{,}
            \PY{p}{]}
            \PY{o}{+}\PY{n}{fina\PYZus{}bias}\PY{p}{(}\PY{p}{)}
            \PY{p}{,}
        \PY{p}{\PYZcb{}}\PY{p}{,}
        \PY{l+s+s1}{\PYZsq{}}\PY{l+s+s1}{sweep\PYZus{}config}\PY{l+s+s1}{\PYZsq{}}\PY{p}{:} \PY{p}{\PYZob{}}
            \PY{n}{q}\PY{p}{:} \PY{p}{\PYZob{}}
                \PY{l+s+s1}{\PYZsq{}}\PY{l+s+s1}{addr}\PY{l+s+s1}{\PYZsq{}}\PY{p}{:} \PY{l+s+sa}{f}\PY{l+s+s1}{\PYZsq{}}\PY{l+s+s1}{gate.Measure.}\PY{l+s+si}{\PYZob{}}\PY{n}{q}\PY{l+s+si}{\PYZcb{}}\PY{l+s+s1}{.}\PY{l+s+si}{\PYZob{}}\PY{n}{cts}\PY{p}{[}\PY{n}{q}\PY{p}{]}\PY{l+s+si}{\PYZcb{}}\PY{l+s+s1}{.frequency}\PY{l+s+s1}{\PYZsq{}}\PY{p}{,}
            \PY{p}{\PYZcb{}}
            \PY{k}{for} \PY{n}{q} \PY{o+ow}{in} \PY{n}{qubits}
        \PY{p}{\PYZcb{}}\PY{p}{,}
        \PY{l+s+s1}{\PYZsq{}}\PY{l+s+s1}{sweep\PYZus{}setting}\PY{l+s+s1}{\PYZsq{}}\PY{p}{:} \PY{p}{\PYZob{}}
\PY{c+c1}{\PYZsh{}             \PYZsq{}repeat\PYZsq{}: np.arange(repeat),}
            \PY{n+nb}{tuple}\PY{p}{(}\PY{n}{qubits}\PY{p}{)}\PY{p}{:} \PY{n+nb}{tuple}\PY{p}{(}\PY{p}{[}
                \PY{n}{sweep\PYZus{}list} \PY{o}{+} \PY{n}{center}\PY{p}{[}\PY{n}{j}\PY{p}{]}
                \PY{k}{for} \PY{n}{j}\PY{p}{,} \PY{n}{i} \PY{o+ow}{in} \PY{n+nb}{enumerate}\PY{p}{(}\PY{n}{qubits}\PY{p}{)}
            \PY{p}{]}\PY{p}{)}\PY{p}{,}
        \PY{p}{\PYZcb{}}\PY{p}{,}
    \PY{p}{\PYZcb{}}
\end{Verbatim}
\end{tcolorbox}

    \begin{tcolorbox}[breakable, size=fbox, boxrule=1pt, pad at break*=1mm,colback=cellbackground, colframe=cellborder]
\prompt{In}{incolor}{27}{\boxspacing}
\begin{Verbatim}[commandchars=\\\{\}]
\PY{n}{qubits} \PY{o}{=} \PY{p}{[}\PY{l+s+s1}{\PYZsq{}}\PY{l+s+s1}{Q0}\PY{l+s+s1}{\PYZsq{}}\PY{p}{,} \PY{l+s+s1}{\PYZsq{}}\PY{l+s+s1}{Q11}\PY{l+s+s1}{\PYZsq{}}\PY{p}{]}

\PY{n}{st}\PY{p}{,} \PY{n}{ed}\PY{p}{,} \PY{n}{sweep\PYZus{}points}\PY{p}{,} \PY{n}{signal} \PY{o}{=} \PY{o}{\PYZhy{}}\PY{l+m+mf}{1e6}\PY{p}{,}\PY{l+m+mf}{1e6}\PY{p}{,} \PY{l+m+mi}{31}\PY{p}{,} \PY{l+s+s1}{\PYZsq{}}\PY{l+s+s1}{remote\PYZus{}iq\PYZus{}avg}\PY{l+s+s1}{\PYZsq{}}

\PY{n}{para\PYZus{}dict} \PY{o}{=} \PY{n}{S21}\PY{p}{(}\PY{n}{qubits}\PY{o}{=}\PY{n}{qubits}\PY{p}{,} \PY{n}{st}\PY{o}{=}\PY{n}{st}\PY{p}{,} \PY{n}{ed}\PY{o}{=}\PY{n}{ed}\PY{p}{,} \PY{n}{sweep\PYZus{}points}\PY{o}{=}\PY{n}{sweep\PYZus{}points}\PY{p}{,} \PY{n}{mode}\PY{o}{=}\PY{l+s+s1}{\PYZsq{}}\PY{l+s+s1}{log}\PY{l+s+s1}{\PYZsq{}}\PY{p}{,} \PY{n}{signal}\PY{o}{=}\PY{n}{signal}\PY{p}{)}
\end{Verbatim}
\end{tcolorbox}

    \begin{tcolorbox}[breakable, size=fbox, boxrule=1pt, pad at break*=1mm,colback=cellbackground, colframe=cellborder]
\prompt{In}{incolor}{17}{\boxspacing}
\begin{Verbatim}[commandchars=\\\{\}]
\PY{n}{task} \PY{o}{=} \PY{n}{general\PYZus{}run\PYZus{}task}\PY{p}{(}\PY{n}{para\PYZus{}dict}\PY{p}{,} \PY{l+m+mi}{1800}\PY{p}{,} \PY{n}{dry\PYZus{}run}\PY{o}{=}\PY{k+kc}{True}\PY{p}{,} \PY{n}{bar}\PY{o}{=}\PY{k+kc}{False}\PY{p}{)}
\end{Verbatim}
\end{tcolorbox}

    \begin{Verbatim}[commandchars=\\\{\}]
S21(11093862863540805035, record\_id=183814)
    \end{Verbatim}

    \begin{tcolorbox}[breakable, size=fbox, boxrule=1pt, pad at break*=1mm,colback=cellbackground, colframe=cellborder]
\prompt{In}{incolor}{25}{\boxspacing}
\begin{Verbatim}[commandchars=\\\{\}]
\PY{n}{task}\PY{o}{.}\PY{n}{plot\PYZus{}prog\PYZus{}frame}\PY{p}{(}\PY{l+m+mi}{0}\PY{p}{,} \PY{n}{start}\PY{o}{=}\PY{l+m+mf}{0e\PYZhy{}6}\PY{p}{,} \PY{n}{stop}\PY{o}{=}\PY{l+m+mf}{6.5e\PYZhy{}6}\PY{p}{,} \PY{n}{raw}\PY{o}{=}\PY{k+kc}{True}\PY{p}{,} \PY{n}{sample\PYZus{}rate}\PY{o}{=}\PY{l+m+mf}{6e9}\PY{p}{,} \PY{n}{keys}\PY{o}{=}\PY{p}{[}\PY{l+s+s1}{\PYZsq{}}\PY{l+s+s1}{M0}\PY{l+s+s1}{\PYZsq{}}\PY{p}{,} \PY{l+s+s1}{\PYZsq{}}\PY{l+s+s1}{M1}\PY{l+s+s1}{\PYZsq{}}\PY{p}{,} \PY{l+s+s1}{\PYZsq{}}\PY{l+s+s1}{Q0}\PY{l+s+s1}{\PYZsq{}}\PY{p}{,} \PY{l+s+s1}{\PYZsq{}}\PY{l+s+s1}{Q11}\PY{l+s+s1}{\PYZsq{}}\PY{p}{]}\PY{p}{)}
\end{Verbatim}
\end{tcolorbox}

    
    \begin{Verbatim}[commandchars=\\\{\}]
<IPython.core.display.Javascript object>
    \end{Verbatim}

    
    
    \begin{Verbatim}[commandchars=\\\{\}]
<IPython.core.display.HTML object>
    \end{Verbatim}

    
    \begin{tcolorbox}[breakable, size=fbox, boxrule=1pt, pad at break*=1mm,colback=cellbackground, colframe=cellborder]
\prompt{In}{incolor}{ }{\boxspacing}
\begin{Verbatim}[commandchars=\\\{\}]
\PY{n}{result} \PY{o}{=} \PY{n}{task}\PY{o}{.}\PY{n}{result}\PY{p}{(}\PY{p}{)}

\PY{n}{fig}\PY{p}{,} \PY{n}{ax} \PY{o}{=} \PY{n}{plt}\PY{o}{.}\PY{n}{subplots}\PY{p}{(}\PY{p}{(}\PY{n+nb}{len}\PY{p}{(}\PY{n}{qubits}\PY{p}{)}\PY{o}{+}\PY{l+m+mi}{4}\PY{p}{)}\PY{o}{/}\PY{o}{/}\PY{l+m+mi}{5}\PY{p}{,} \PY{l+m+mi}{5}\PY{p}{,} \PY{n}{figsize}\PY{o}{=}\PY{p}{[}\PY{l+m+mi}{8}\PY{p}{,} \PY{p}{(}\PY{n+nb}{len}\PY{p}{(}\PY{n}{qubits}\PY{p}{)}\PY{o}{+}\PY{l+m+mi}{4}\PY{p}{)}\PY{o}{/}\PY{o}{/}\PY{l+m+mi}{5}\PY{o}{*}\PY{l+m+mf}{1.6}\PY{p}{]}\PY{p}{)}
\PY{n}{ax} \PY{o}{=} \PY{n}{ax}\PY{o}{.}\PY{n}{flatten}\PY{p}{(}\PY{p}{)}
\PY{n}{fig}\PY{o}{.}\PY{n}{suptitle}\PY{p}{(}\PY{l+s+sa}{f}\PY{l+s+s2}{\PYZdq{}}\PY{l+s+si}{\PYZob{}}\PY{n}{task}\PY{o}{.}\PY{n}{name}\PY{l+s+si}{\PYZcb{}}\PY{l+s+s2}{ id=}\PY{l+s+si}{\PYZob{}}\PY{n}{task}\PY{o}{.}\PY{n}{record\PYZus{}id}\PY{l+s+si}{\PYZcb{}}\PY{l+s+s2}{\PYZdq{}}\PY{p}{)}

\PY{k+kn}{from}\PY{+w}{ }\PY{n+nn}{qos\PYZus{}tools}\PY{n+nn}{.}\PY{n+nn}{analyzer}\PY{n+nn}{.}\PY{n+nn}{tools}\PY{+w}{ }\PY{k+kn}{import} \PY{n}{get\PYZus{}normalization}\PY{p}{,} \PY{n}{get\PYZus{}convolve\PYZus{}arg}

\PY{n}{cali} \PY{o}{=} \PY{p}{\PYZob{}}\PY{p}{\PYZcb{}}

\PY{k}{for} \PY{n}{i}\PY{p}{,} \PY{n}{q} \PY{o+ow}{in} \PY{n+nb}{enumerate}\PY{p}{(}\PY{n}{qubits}\PY{p}{)}\PY{p}{:}
    \PY{n}{flag}\PY{p}{,} \PY{n}{ans} \PY{o}{=} \PY{n}{get\PYZus{}convolve\PYZus{}arg}\PY{p}{(}\PY{n}{x}\PY{o}{=}\PY{n}{result}\PY{p}{[}\PY{l+s+s1}{\PYZsq{}}\PY{l+s+s1}{index}\PY{l+s+s1}{\PYZsq{}}\PY{p}{]}\PY{p}{[}\PY{n}{q}\PY{p}{]}\PY{p}{[}\PY{p}{:}\PY{p}{]}\PY{p}{,} \PY{n}{y}\PY{o}{=}\PY{n}{np}\PY{o}{.}\PY{n}{abs}\PY{p}{(}\PY{n}{result}\PY{p}{[}\PY{n}{signal}\PY{p}{]}\PY{p}{[}\PY{p}{:}\PY{p}{,} \PY{n}{i}\PY{p}{]}\PY{p}{)}\PY{p}{,} 
                                 \PY{n}{ax}\PY{o}{=}\PY{n}{ax}\PY{p}{[}\PY{n}{i}\PY{p}{]}\PY{p}{,} \PY{n}{ext}\PY{o}{=}\PY{l+s+s1}{\PYZsq{}}\PY{l+s+s1}{min}\PY{l+s+s1}{\PYZsq{}}\PY{p}{)}
    \PY{k}{if} \PY{n}{flag}\PY{p}{:}
        \PY{n}{ax}\PY{p}{[}\PY{n}{i}\PY{p}{]}\PY{o}{.}\PY{n}{axvline}\PY{p}{(}\PY{n}{x}\PY{o}{=}\PY{n}{result}\PY{p}{[}\PY{l+s+s1}{\PYZsq{}}\PY{l+s+s1}{index}\PY{l+s+s1}{\PYZsq{}}\PY{p}{]}\PY{p}{[}\PY{n}{q}\PY{p}{]}\PY{p}{[}\PY{n}{np}\PY{o}{.}\PY{n}{argmin}\PY{p}{(}\PY{n}{get\PYZus{}normalization}\PY{p}{(}\PY{n}{np}\PY{o}{.}\PY{n}{abs}\PY{p}{(}\PY{n}{result}\PY{p}{[}\PY{n}{signal}\PY{p}{]}\PY{p}{[}\PY{p}{:}\PY{p}{,} \PY{n}{i}\PY{p}{]}\PY{p}{)}\PY{p}{)}\PY{p}{)}\PY{p}{]}\PY{p}{,} \PY{n}{c}\PY{o}{=}\PY{l+s+s1}{\PYZsq{}}\PY{l+s+s1}{k}\PY{l+s+s1}{\PYZsq{}}\PY{p}{,} \PY{n}{ls}\PY{o}{=}\PY{l+s+s1}{\PYZsq{}}\PY{l+s+s1}{\PYZhy{}\PYZhy{}}\PY{l+s+s1}{\PYZsq{}}\PY{p}{)}
\PY{c+c1}{\PYZsh{}         cali[f\PYZsq{}gate.Measure.\PYZob{}q\PYZcb{}.params.frequency\PYZsq{}] = result[\PYZsq{}index\PYZsq{}][q][np.argmin(get\PYZus{}normalization(np.abs(result[signal][:, i])))]}
        \PY{n}{cali}\PY{p}{[}\PY{l+s+sa}{f}\PY{l+s+s1}{\PYZsq{}}\PY{l+s+s1}{gate.Measure.}\PY{l+s+si}{\PYZob{}}\PY{n}{q}\PY{l+s+si}{\PYZcb{}}\PY{l+s+s1}{.params.frequency}\PY{l+s+s1}{\PYZsq{}}\PY{p}{]} \PY{o}{=} \PY{n}{ans}
\PY{c+c1}{\PYZsh{}     ax[i].plot(result[\PYZsq{}index\PYZsq{}][q][:], (np.abs(result[signal][:, i]))/1e8, \PYZsq{}.\PYZhy{}\PYZsq{})}
\PY{c+c1}{\PYZsh{}     cali[f\PYZsq{}gate.Measure.\PYZob{}q\PYZcb{}.params.frequency\PYZsq{}] = result[\PYZsq{}index\PYZsq{}][q][np.argmin(np.abs(result[signal][:, i]))]}
\PY{c+c1}{\PYZsh{}     ax[i].axvline(x=cali[f\PYZsq{}gate.Measure.\PYZob{}q\PYZcb{}.params.frequency\PYZsq{}], c=\PYZsq{}r\PYZsq{})}
    \PY{n}{ax}\PY{p}{[}\PY{n}{i}\PY{p}{]}\PY{o}{.}\PY{n}{set\PYZus{}title}\PY{p}{(}\PY{n}{q}\PY{p}{,} \PY{n}{fontsize}\PY{o}{=}\PY{l+m+mi}{8}\PY{p}{)}
        
\PY{n}{fig}\PY{o}{.}\PY{n}{tight\PYZus{}layout}\PY{p}{(}\PY{p}{)}
\PY{n}{fig}\PY{o}{.}\PY{n}{show}\PY{p}{(}\PY{p}{)}
\end{Verbatim}
\end{tcolorbox}

    \begin{tcolorbox}[breakable, size=fbox, boxrule=1pt, pad at break*=1mm,colback=cellbackground, colframe=cellborder]
\prompt{In}{incolor}{ }{\boxspacing}
\begin{Verbatim}[commandchars=\\\{\}]
\PY{n}{kernel}\PY{o}{.}\PY{n}{update\PYZus{}parameters}\PY{p}{(}\PY{n}{cali}\PY{p}{)}
\PY{n}{plt}\PY{o}{.}\PY{n}{close}\PY{p}{(}\PY{l+s+s1}{\PYZsq{}}\PY{l+s+s1}{all}\PY{l+s+s1}{\PYZsq{}}\PY{p}{)}
\end{Verbatim}
\end{tcolorbox}

    \subsection{R}\label{r}

    \begin{tcolorbox}[breakable, size=fbox, boxrule=1pt, pad at break*=1mm,colback=cellbackground, colframe=cellborder]
\prompt{In}{incolor}{ }{\boxspacing}
\begin{Verbatim}[commandchars=\\\{\}]

\end{Verbatim}
\end{tcolorbox}

    \begin{tcolorbox}[breakable, size=fbox, boxrule=1pt, pad at break*=1mm,colback=cellbackground, colframe=cellborder]
\prompt{In}{incolor}{ }{\boxspacing}
\begin{Verbatim}[commandchars=\\\{\}]

\end{Verbatim}
\end{tcolorbox}

    \begin{tcolorbox}[breakable, size=fbox, boxrule=1pt, pad at break*=1mm,colback=cellbackground, colframe=cellborder]
\prompt{In}{incolor}{ }{\boxspacing}
\begin{Verbatim}[commandchars=\\\{\}]

\end{Verbatim}
\end{tcolorbox}

    \begin{tcolorbox}[breakable, size=fbox, boxrule=1pt, pad at break*=1mm,colback=cellbackground, colframe=cellborder]
\prompt{In}{incolor}{ }{\boxspacing}
\begin{Verbatim}[commandchars=\\\{\}]

\end{Verbatim}
\end{tcolorbox}


    % Add a bibliography block to the postdoc
    
    
    
\end{document}
